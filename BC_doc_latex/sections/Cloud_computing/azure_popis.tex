\begin{document}

\subsubsection{Azure}
Je to cloudová platoforma, ktorá prevádzkuje integráciu cloudových služieb. Pomocou použitia Azure môžme meniť a prispôsobiť aplikáciu podľa potrieb zákazníka. Pre zjednodušenie, manažovania aplikácií je vytvorený Azure portal. Služby podľa odvetví sú rozdelené do týchto kategórií:
\begin{multicols}{3}

    \begin{itemize}
        \item AI + Machine Learning
        \item Analytics
        \item Blockchain
        \item Compute
        \item Containers
        \item Databases
        \item Developer Tools
        \item DevOps
        \item Hybrid
        \item Identity
        \item Integration
        \item Internet of Things
        \item Management \\and Governance
        \item Media
        \item Migration
        \item Mixed Reality
        \item Mobile
        \item Networking
        \item Security
        \item Storage
        \item Web
        \item Windows Virtual\\ Desktop
    
    \end{itemize}
 %ked tak pridat linky   
\end{multicols}
Niekoľko odporúčaných služieb vhodných pre použitie aplikácie:
\begin{description}
    \item[Azure App Service] je vhodné pre vytvorenie webových aplikácií, backendy mobilných aplikácií a API a migráciu už existujúcich aplikácií.
    Platforma poskytuje autentifikáciu pomocou sociálnych poskytovateľov, škálovateľnosť podľa zaťaženia, testovanie a nasadenie nepretržitých kontajnerov.
    \item[Azure Virtual Machines] je Infrastructure as a Service (IaaS) služba. Je možné nasadenie na Linuxovom alebo Windowsovom  virtuálnom stroji. Klient je sa stará o všetky požiadavky na virtuálnom stroji (inštalácia softvéru, konfigurácia, údržba a aktualizácia. Je možné spustiť množstvo záťažových testov. Odporúčané použitie pri potrebe kontroly nad celou infraštruktúrou.
    \item[Azure Functions (serverless)]  - vhodné pre priame spustenie kódu bez nastavovania infraštruktúry. Na použitie je množstvo programovacích jazykov ako C\#, F\#, Node.js, Python, alebo PHP. Platí sa iba za čas, v ktorom bol vykonávaný kód. Vhodné pri skúšaní len časti aplikácie.
    \item[Azure Service Fabric] je platforma distribuovaných systémov. Umožňuje budovanie package a nasadenie mikroservisov. Poskytuje schopnosti zabezpečenia služieb, nasadenia, monitorovania, aktualizácie a odstránenia. Aplikácie sú spúšťané na celých adresároch, je možné štartovať množstvo menších strojov podľa potreby. Vhodné pri použití mikroservisovej architektúry.
    \item[Docker podpora] umožňuje použitie kontajnerov Docker, čo je určitá forma virtualizácie operačných systémov. Pomocou Dockera je možné vytvoriť aplikáciu viac účinnejšie a predvídateľnejšie. Sú dostupné bežné Docker nástroje pre manažovanie kontajnerov. Použitie kontajnerov je možné viacerými spôsobmi:
    \begin{description}
        \item[Azure Docker VM extension:]
            Nastavenie virtuálneho stroja s Docker nástrojmi ako Docker host
        \item[Azure Kubernetes Service:]
        Umožňuje vytvoriť, nastavať a manažovať klaster virtuálnych strojov, ktoré sú prednastavené na spustenie kontajnerizovaných aplikácií. Vhodné pri použití prostredia pripraveného pre produkcie alebo pri použití technológie Docker Swarm cluster.
        \item[Docker Machine:] Umožňuje použitie a manažovanie deamonu Docker Engine na virtuálnom stroji. Vhodné pre rýchle prototypovanie aplikácie na jednom vytvorenom Docker hoste.
        \item[Custom Docker image for App Service] umožňuje použitie Docker kontajnera z registra kontajnerov alebo užívateľského kontajneru pri nasadení web-aplikácie na linuxe.
    \end{description}
\end{description}
\paragraph{Autentifikácia} je možná buď pomocou Azure Active Directory (Azure AD), kde sa využívajú tokeny a ďalšie služby (SSO - single-sign on, Graph-based data) alebo pomocou autentifikácie služby aplikácie.% ???

Po spustení aplikácie je možné ju monitorovať pomocou monitorovacích funkcií. Je možné používanie populárnych DevOps nástrojov (Jenkins,GitHub,Puppet,Chef,TeamCity,Ansible,Azure, DevOps). Azure platforma je rozdelená do niekoľkých dáta centier po celom svete čím sa docieli zníženie latencie aplikácie. Manažovanie Azure resources je možné pomocou prostredia Azure portal, Command-line rozhrania alebo PowerShellu. Je využiteľná aj množina REST APIs, ktorými je tvorený Azure pre podporu Azure portalu.
Azure Resource Manager umožňuje triediť Azure Resources do skupín pre lepší manažment.\cite{azure}
\paragraph{Klientské účty a poplatky: }
Približné mesačne náklady je možné vypočítať cez virtuálnu kalkulačku na stránke \url{https://azure.microsoft.com/en-us/pricing/calculator/}. Pre študenta je možné využiť kredit 100\$ po dobu 12 mesiacov.\cite{azure_student}



\end{document}