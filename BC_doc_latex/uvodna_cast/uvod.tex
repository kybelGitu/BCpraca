\documentclass[./main.tex]{subfiles} %kvoli pridani preambluly

\begin{document}

\newpage

\section*{Úvod}
 \addcontentsline{toc}{section}{Úvod}%
 
 %11)	V úvode autor stručne a výstižne charakterizuje stav poznania alebo praxe v oblasti, ktorá je predmetom ZP a oboznamuje s významom riešenej problematiky. 
V súčasnom technologickom trende dochádza k aplikácií nezávislými elektronickými zariadeniami, ktoré nevyžadujú takmer žiadnu obsluhu, majú nízku spotrebu a jednoduchý tvar s malým objemom.
 Takéto zariadenia sú schopné plniť svoju funkciu aj niekoľko rokov. Bez výmeny batérie, ktorá je zvyčajne menšia. Na komunikáciu sa využívajú protokoly a siete, ktoré nevyžadujú veľkú spotrebu. Pri spustení takéhoto riešenia treba myslieť aj na zabezpečenie siete a ukladanie dát. Pri meraní vo fyzikálnom laboratóriu, kde sú spustené dlhodobé meranie je nutné sledovať dátový úložný priestor pre možnú poruchu, alebo zaplnenie disku. Je vhodné namerané dáta ukladať aj mimo laboratória, ak by došlo k zlyhaniu systému ako celku. Môže to byť server alebo cloudové riešenie, kde nie je nutné starať sa o celú infraštruktúru.
%priemyselny sektor , domacnosti
%princip 
%technologie
% 
\end{document}


