\begin{document}

\subsection{Ekvivalentný dávkový príkon gamma žiarenia}
V záujme bezpečnosti bola na území SR vytvorená radiačná monitorovacia sieť. Namerané údaje z tejto siete sú zdielané s okolitými štátmi kvôli včasnému varovaniu v dôsledku havárie. Veličina, ktorá sa meria kvôli včasnému varovaniu je príkon absorbovanej látky, ktorý slúži pre stanovenie príkonu dávkového ekvivalentu gama žiarenia v ovzduší. 
Základné informácie o žiarení a rádioaktivite\cite{radioaktivita_shmu}:
\begin{table}[h!]
    \centering
    \begin{tabular}{|c|c|}
        \hline
         &  Žiarenie prenášajúce $E$ vo forme \\
        \textbf{Ionizujúce žiarenie} &častíc alebo elektromagnetických vĺn \\
         &s $\lambda$  do $100 nm$ alebo s $f\,=\,3.1015 Hz$\\
         \hline
        \textbf{Ožiarenie} & Vystavenie pôsobeniu ionizujúceho žiarenia\cite{radioaktivita_pdf} \\
        \hline
        \textbf{Príkon dávkového ekvivalentu} & Súčin absorbovanej dávky, \\
         & akostného faktora a ďalších modifikujúcich faktorov\\
        \hline
         &   Absorbovaná dávka x akostný súčiniteľ \\
         \textbf{Dávkový ekvivalent}& jednotka: sievert (Sv),\\
         & v sieti sa meria ukazovateľ v $nSv/h$\\
         \hline
          & Energia odovzdaná ionizujúcim\\
          \textbf{Absorbovaná dávka} & žiarením látke príslušne \\
           & malého elementu objemu, \\
           & delená hmotnosťou tohto elementu objemu.\\
           \hline
    \end{tabular}
    \caption{Rádioaktivita - dávkový  príkon}
    \label{tab:radioaktivita}
\end{table}
% Dávkový ekvivalent = absorbovaná dávka x akostný súčiniteľ
\end{document}